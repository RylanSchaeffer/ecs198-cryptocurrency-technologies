\title{\bf Lecture 3 - Centralized Cryptocurrencies\\}
\author{\bf Rylan Schaeffer and Vincent Yang\\}
\date{\bf \today \\}

\documentclass{article}
\renewcommand{\thesubsection}{\thesection.\alph{subsection}}
\usepackage{enumerate}
\usepackage{hyperref}
\usepackage{amsmath}
\usepackage{graphicx}
\setlength{\oddsidemargin}{0in}
\setlength{\evensidemargin}{0in}
\setlength{\textheight}{9in}
\setlength{\textwidth}{6.5in}
\setlength{\topmargin}{-0.5in}

\begin{document}
\maketitle

Note: This lecture is based on Princeton University's BTC-Tech: Bitcoin and Cryptocurrency Technologies Spring 2015 course.


\section*{Centralized Banking}
\begin{itemize}
  \item Centralize: to concentrate under a single authority
  \item Centralized banking means there is a single institution that manages supply, inflation, and interest.
\end{itemize}
\section*{Advantages to Centralization}
\begin{itemize}
  \item Automation:
    \subitem Easily manage a large number of keys e.g. Mastercard Europe
    \subitem Maintain secure infrastructure and improve operations/efficiency
  \item Centralized Monitoring:
    \subitem Record everything that happens easily; brings transparency
  \item Centralized Policy
  \item Easily update and track keys
  \item Easily update cryptographic schemes - swap out algorithms
\end{itemize}

\section*{CentralizedCoin}
\begin{itemize}
  \item I can generate coins, and give them unique ID's. I also sign these coins.
  \item I can pass them to anyone else - I sign the transaction; recipient can prove it's valid because it has my signature.\\
    Recipient can sign to pass to someone else. 
  \item Chain of hash pointers can be used to follow it back. = verify
  \item Double Spending Problem
  \item Only I can write on the chain - everything has to pass through me
  \item This is centralized; how do you trust me? 
\end{itemize}

\section*{Centralized Cryptocurrencies}
\begin{itemize}
  \item E-Gold (1996)
    \subitem Operated by Gold and Silver Reserve inc 
    \subitem Let users open an account denominated in gold; could make instant transfers
    \subitem Grew to 5 million accounts; processing over 2 billion a year
    \subitem "e-Gold Special Purpose Trust" - actually held the gold; could see gold bars with serial numbers per acct.
    \subitem Hackers used flaws in Microsoft Windows OS's and phishing to compromise millions of e-gold accounts
    \subitem People thought it was \emph{anonymous}, but really it was \emph{pseudonymous}. Law enforcement identified many.
    \subitem Ponzi schemes via. eBay 
    \subitem Patriot Act, after Sept 11, made operating a money transmitter business without a state money transmitter license a federal crime.
    \subitem Taken down 2007-2013; inability to provide reliable user identification and cut off illegal activity
    \subitem PayPal has done a better job, but still has to deal with the same problems.
    \subitem KYC - process of verifying clients' identity
  \item Liberty Reserve
    \subitem Shut down, also by Patriot Act, in May 2013. 
  \item E-Gold and Liberty Reserve were popularly used for money laundering and shut down
  \item Can be shut down by the government at any time
  \item DigiCash by Chaum 1990
    \subitem Store money as data on your computer
    \subitem Transfer anonymously
    \subitem Lacked decentralization; the company's servers were used
    \subitem Went bankrupt in 1998
\end{itemize}
Source: \url{http://people.dsv.su.se/~matei/courses/IK2001_SJE/Chaum90.pdf}\\
Source: \url{http://blog.koehntopp.de/uploads/Chaum.BlindSigForPayment.1982.PDF}

\end{document}
